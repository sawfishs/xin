\documentclass[a4paper]{article}

\usepackage{xeCJK}
\usepackage{geometry}%页边距
\geometry{left=2.6cm,right=2.6cm,top=2.6cm,bottom=2.6cm}
\setlength{\parindent}{3em}

\renewcommand{\CJKglue}{\hskip 1pt plus 0.04\baselineskip} %字间距
\linespread{1.6}%行间距
\setlength{\parskip}{1.1 \baselineskip}%段落间距

\setCJKmainfont{Adobe Song Std}
\newcommand{\sihao}{\fontsize{14pt}{\baselineskip}\selectfont}

\title{般若波罗蜜多心经}
\author{}
\date{}

\begin{document}
\maketitle
\vspace{-5em}

\sihao{
观自在菩萨,行深般若波罗蜜多时,照见五蕴皆空,度一切苦厄。


舍利子,色不异空,空不异色,色即是空,空即是色,受想行识亦复如是。


舍利子,是诸法空相,不生不灭,不垢不净,不增不减,是故空中无色,无受想行识,无眼耳鼻舌身意,无色声香味触法,无眼界 乃至无意识界,无无明亦无无明尽,乃至无老死,亦无老死尽,无苦集灭道,无智亦无得。


以无所得故,菩提萨埵,依般若波罗蜜多故,心无挂碍,无挂碍故,无有恐怖,远离颠倒梦想,究竟涅槃。


三世诸佛,依般若波罗蜜多故,得阿耨多罗三藐三菩提。


故知般若波罗蜜多,是大神咒,是大明咒,是无上咒,是无等等咒,能除一切苦,真实不虚。


故说般若波罗蜜多咒,即说咒曰:揭谛揭谛,波罗揭谛,波罗僧揭谛,菩提萨婆诃。
}

\end{document}
